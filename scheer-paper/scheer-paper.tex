\documentclass[letterpaper]{article}

\usepackage{aaai}
\usepackage{times}
\usepackage{helvet}
\usepackage{courier}
\usepackage[pdftex,dvipsnames]{xcolor}
\usepackage{xargs}
\usepackage{hyperref}

\usepackage[colorinlistoftodos,prependcaption,textsize=tiny]{todonotes}
\newcommandx{\unsure}[2][1=]{\todo[linecolor=red,backgroundcolor=red!25,textcolor=black,bordercolor=red,#1]{#2}}
\newcommandx{\change}[2][1=]{\todo[linecolor=blue,backgroundcolor=blue!25,bordercolor=blue,#1]{#2}}
\newcommandx{\info}[2][1=]{\todo[linecolor=OliveGreen,backgroundcolor=OliveGreen!25,bordercolor=OliveGreen,#1]{#2}}
\newcommandx{\improvement}[2][1=]{\todo[linecolor=Plum,backgroundcolor=Plum!25,bordercolor=Plum,#1]{#2}}
\newcommandx{\thiswillnotshow}[2][1=]{\todo[disable,#1]{#2}}

\frenchspacing
\setlength{\pdfpagewidth}{8.5in}
\setlength{\pdfpageheight}{11in}
\pdfinfo{
/Title (Finding Parallel Regions with Temporal Planning)
/Author (Claudio Scheer)}
\setcounter{secnumdepth}{0}

\begin{document}

\title{Finding Parallel Regions with Temporal Planning}
\author{Claudio Scheer\\
    Master's Degree in Computer Science\\
    Pontifical Catholic University of Rio Grande do Sul - PUCRS\\
    Porto Alegre - RS, Brazil\\
    claudio.scheer@edu.pucrs.br\\
}
\maketitle

\begin{abstract}
    \begin{quote}
        Abstract.
    \end{quote}
\end{abstract}

\noindent Introduction.


\section{Bibliography}

\subsection{Temporal Plannign}

According to \cite{DBLP:series/synthesis/2019Haslum}, actions in temporal planning have a duration. Therefore, the planner will try to find a schedule in which some actions can be executed in parallel.

There are different approaches that can be used to formalize temporal actions with PDDL. In this paper, I used \texttt{:durative-actions}. This action is represented in four sections, as listed below.

\begin{itemize}
    \item \texttt{:parameters}: parameters needed to execute the action;
    \item \texttt{:duration}: time the action takes to run;
    \item \texttt{:condition}: conditions that need to be respected to apply the effects;
    \item \texttt{:effect}: effects that will be applied to the state;
\end{itemize}

The sections \texttt{:condition} and \texttt{:effect} are separated in three categories: \texttt{at start}, \texttt{over all} and \texttt{at end}. As described by \cite{DBLP:series/synthesis/2019Haslum}, these categories represent the conditions and effects used at each stage of the action. The \texttt{at start} statements are used when starting the action. The \texttt{over all} statements are used during the time the action is being executed. The \texttt{at end} statements are used at the end of the action.

\improvement[inline]{The following section may not be necessary.}
I must mention that \texttt{:durative-actions} can be translated into instantaneous actions. Here is the paper: \cite{DBLP:conf/ecai/ScalaHTR16}.


\subsection{STP}

I used the STP (Simultaneous Temporal Planner) planner to find the best temporal plan. STP uses a modified version of the Fast Downward \cite{Helmert_2006} planner that can generate a temporal plan. The version used in this paper is provided by the Artificial Intelligence and Machine Learning Group - Universitat Pompeu Fabra\footnote{\href{https://github.com/aig-upf}{https://github.com/aig-upf}}.

The STP planner needs to receive as a parameter the maximum number of actions that can be executed at the same time. When finding parallel regions, this parameter is a problem, because we do not know how many instructions can be executed in parallel. Therefore, in some cases, it is necessary to test different values for this paramenter.


\section{Formalization}

\info[inline]{Describe here how I did the formalization.}


\section{Results}

\info[inline]{Describe here the results.}


\bibliographystyle{aaai}
\bibliography{references}

\end{document}