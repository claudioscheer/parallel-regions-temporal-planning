\documentclass[letterpaper]{article}

\usepackage{aaai}
\usepackage{times}
\usepackage{helvet}
\usepackage{courier}

\frenchspacing
\setlength{\pdfpagewidth}{8.5in}
\setlength{\pdfpageheight}{11in}
\pdfinfo{
/Title (Finding Parallel Regions with Temporal Planning)
/Author (Claudio Scheer)}
\setcounter{secnumdepth}{0}

\begin{document}

\title{Finding Parallel Regions with Temporal Planning}
\author{Claudio Scheer\\
    Master's Degree in Computer Science\\
    Pontifical Catholic University of Rio Grande do Sul - PUCRS\\
    Porto Alegre - RS, Brazil\\
    claudio.scheer@edu.pucrs.br\\
}
\maketitle

\begin{abstract}
    \begin{quote}
        Abstract.
    \end{quote}
\end{abstract}

\noindent Introduction.


\section{Bibliography}

\subsection{Temporal Plannign}

According to \cite{DBLP:series/synthesis/2019Haslum}, actions in temporal planning have a duration. Therefore, the planner will try to find a schedule in which some actions can be executed in parallel.

There are different approaches that can be used to formalize temporal actions with PDDL. In this paper, I used \texttt{:durative-actions}. This action is represented in four sections, as listed below.

\begin{itemize}
    \item \texttt{:parameters}: parameters needed to execute the action;
    \item \texttt{:duration}: time the action takes to run;
    \item \texttt{:condition}: conditions that need to be respected to apply the effects;
    \item \texttt{:effect}: effects that will be applied to the state;
\end{itemize}

The sections \texttt{:condition} and \texttt{:effect} are separated in three categories: \texttt{at start}, \texttt{over all} and \texttt{at end}. As described by \cite{DBLP:series/synthesis/2019Haslum}, these categories represent the conditions and effects used at each stage of the action. The \texttt{at start} statements are used when starting the action. The \texttt{over all} statements are used during the time the action is being executed. The \texttt{at end} statements are used at the end of the action.


\subsection{Fast Downward}

I used the Fast Downward planner to find best temporal plan. Originaly, the paper form \cite{Helmert_2006} does not state that Fast Downward planner support temporal planning.


\bibliographystyle{aaai}
\bibliography{references}

\end{document}