\documentclass{beamer}
\usepackage[utf8]{inputenc}

\usepackage{utopia} %font utopia imported
\usepackage{tabularx}

\usetheme{Madrid}
\usecolortheme{default}

%------------------------------------------------------------
%This block of code defines the information to appear in the
%Title page
\title[Parallel Regions with PDDL]
{Parallel Regions using a PDDL Formalization}

\subtitle{PDDL domain as a compiler}

\author[Claudio Scheer]
{Claudio~Scheer\inst{1}}

\institute[PUCRS]
{
  \inst{1}%
  Master's Degree in Computer Science\\
  Pontifical Catholic University of Rio Grande do Sul - PUCRS
}

\date[May 2020]
{Paper Proposal, May 2020}

% \logo{\includegraphics[height=1.5cm]{lion-logo.jpg}}

%End of title page configuration block
%------------------------------------------------------------



%------------------------------------------------------------
%The next block of commands puts the table of contents at the 
%beginning of each section and highlights the current section:

\AtBeginSection[]
{
  \begin{frame}
    \frametitle{Table of Contents}
    \tableofcontents[currentsection]
  \end{frame}
}
%------------------------------------------------------------


\begin{document}

\frame{\titlepage}

%---------------------------------------------------------

\begin{frame}
  \frametitle{Table of Contents}
  \tableofcontents
\end{frame}

%---------------------------------------------------------

\section{Problem}

\begin{frame}
  \frametitle{Finding parallel regions}

  \begin{itemize}
    \item It takes a lot of time;
    \item It will cost money.
  \end{itemize}
\end{frame}

\begin{frame}
  \frametitle{Common approaches}
  Static analysis of the source code:

  \begin{itemize}
    \item<1-> loops detection;
    \item<2-> variable dependencies;
    \item<3-> identifying whether the arguments are read or written.
  \end{itemize}
\end{frame}

%---------------------------------------------------------

\section{Proposed approach}

\begin{frame}
  \frametitle{PDDL domain}

  PDDL domain will work as a compiler. \pause

  \begin{itemize}
    \item Instructions support:
          \begin{itemize}
            \item arithmetic and binary operators;
            \item functions;
            \item loop.
          \end{itemize}
  \end{itemize}
\end{frame}

\begin{frame}
  \frametitle{PDDL problem}

  Source code will be mapped to a set of predicates. \pause

  \begin{itemize}
    \item Goal:
          \begin{itemize}
            \item execute all instructions;
            \item must run in the correct order.
          \end{itemize}
  \end{itemize}
\end{frame}

\begin{frame}
  \frametitle{Bottleneck}

  \begin{alertblock}{Bad}
    Creating a set of predicates from source code will not be easy.
  \end{alertblock}

  \begin{block}{Good}
    It can be easily automated.
  \end{block}
\end{frame}

%---------------------------------------------------------

\section{Results evaluation}

\begin{frame}
  \frametitle{Results evaluation}

  \begin{itemize}
    \item Decode planner output to source code;
    \item Validate the parallel execution;
    \item Was the execution time shorter?
          \begin{itemize}
            \item I should probably test a big problem;
            \item I may not have enough time.
          \end{itemize}
  \end{itemize}
\end{frame}

%---------------------------------------------------------

\section{Questions/Ideas}

\begin{frame}
  \frametitle{Questions/Ideas}

  \begin{enumerate}
    \item<1-> Is the compiler domain capable of handling fluent variables and predicates?
    \item<2-> Is the compiler domain capable of performing operations with strings?
    \item<3-> Which planners should I test the compiler domain on?
    \item<4-> How does a planner find a parallel region?
    \item<5-> Can I set a weight for the planner to get regions that are really worth running in parallel?
  \end{enumerate}
\end{frame}

%---------------------------------------------------------

\section{Schedule}

\begin{frame}
  \frametitle{Schedule}

  \begin{center}
    \begin{tabularx}{0.9\textwidth}{
        | >{\raggedright\arraybackslash}X
        | >{\raggedright\arraybackslash}X
        | >{\raggedright\arraybackslash}X|}
      \hline
      \textbf{Task}                 & \textbf{Start} & \textbf{End} \\
      \hline
      Understand better compilers   & 06-01-2020     & 06-03-2020   \\
      \hline
      Support sum instruction       & 06-03-2020     & 06-07-2020   \\
      \hline
      Support proposed instructions & 06-08-2020     & 06-15-2020   \\
      \hline
      Evaluate results              & 06-16-2020     & 06-20-2020   \\
      \hline
      Write paper                   & 06-20-2020     & 06-25-2020   \\
      \hline
    \end{tabularx}
  \end{center}
\end{frame}

%---------------------------------------------------------

\section{Conclusion}

\begin{frame}
  \frametitle{Conclusion}

  \begin{itemize}
    \item<1-> This is not a conventional approach;
    \item<2-> If the results are positives, the approach may reduce the amount of time to find parallel regions.
  \end{itemize}
\end{frame}

\end{document}